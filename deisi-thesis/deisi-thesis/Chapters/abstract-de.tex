%!TEX root = ../template.tex
%%%%%%%%%%%%%%%%%%%%%%%%%%%%%%%%%%%%%%%%%%%%%%%%%%%%%%%%%%%%%%%%%%%%
%% abstrac-de.tex
%% NOVA thesis document file
%%
%% Abstract in English
%%%%%%%%%%%%%%%%%%%%%%%%%%%%%%%%%%%%%%%%%%%%%%%%%%%%%%%%%%%%%%%%%%%%

\typeout{NT FILE abstrac-de.tex}%

\textbf{Dies ist eine „Google Translate“-Übersetzung der englischen Version! Fixes und Korrekturen sind willkommen!}

Unabhängig von der Sprache, in der die Dissertation verfasst ist, gibt es in der Regel mindestens zwei Abstracts: einen Abstract in der gleichen Sprache wie der Haupttext und einen Abstract in einer anderen Sprache.

Die Reihenfolge der Abstracts variiert je nach Schule. Wenn Ihre Schule spezielle Vorschriften bezüglich der Reihenfolge der Abstracts hat, wird die Vorlage \gls{novathesis} (\LaTeX) diese respektieren. Andernfalls ist die Standardregel in der \gls{novathesis}-Vorlage, zuerst die Zusammenfassung in \emph{der gleichen Sprache wie der Haupttext} zu haben, und dann die Zusammenfassung in \emph{der anderen Sprache}. Wenn die Dissertation beispielsweise auf Portugiesisch verfasst ist, ist die Reihenfolge der Zusammenfassungen zuerst Portugiesisch und dann Englisch, gefolgt vom Haupttext auf Portugiesisch. Wenn die Dissertation in englischer Sprache verfasst ist, ist die Reihenfolge der Abstracts zuerst englisch und dann portugiesisch, gefolgt vom Haupttext in englischer Sprache.
%
Diese Reihenfolge kann jedoch angepasst werden, indem der Datei \verb+5_packages.tex+ eines der folgenden Elemente hinzugefügt wird.

\begin{verbatim}
    \ntsetup{abstractorder={<LANG_1>,...,<LANG_N>}}
    \ntsetup{abstractorder={<MAIN_LANG>={<LANG_1>,...,<LANG_N>}}}
\end{verbatim}

Verwenden Sie zum Beispiel für ein auf Deutsch verfasstes Hauptdokument mit Zusammenfassungen auf Deutsch, Englisch und Italienisch (in dieser Reihenfolge):
\begin{verbatim}
    \ntsetup{abstractorder={de={de,en,it}}}
\end{verbatim}

Inhaltlich sollten die Abstracts eine Seite nicht überschreiten und können folgende Fragen beantworten (unbedingt an die Gepflogenheiten Ihres Fachgebietes anpassen):

\begin{itemize}
  \item Was ist das Problem?
  \item Warum ist es interessant?
  \item Was ist die Lösung?
  \item Was folgt aus der Lösung?
\end{itemize}

% Palavras-chave do resumo em Inglês
\begin{keywords}
Schlüsselwort 1, Schlüsselwort 2, Schlüsselwort 3
\end{keywords}
