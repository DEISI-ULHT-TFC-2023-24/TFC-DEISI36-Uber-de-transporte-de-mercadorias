%!TEX root = ../template.tex
%%%%%%%%%%%%%%%%%%%%%%%%%%%%%%%%%%%%%%%%%%%%%%%%%%%%%%%%%%%%%%%%%%%%
%% abstract-it.tex
%% NOVA thesis document file
%%
%% Abstract in Italian
%%%%%%%%%%%%%%%%%%%%%%%%%%%%%%%%%%%%%%%%%%%%%%%%%%%%%%%%%%%%%%%%%%%%

\typeout{NT FILE abstract-it.tex}%


\textbf{¡Esta es una traducción de “Google Translate” de la versión en inglés! ¡Reparaciones y correcciones son bienvenidas!}

Independientemente del idioma en el que esté escrita la disertación, generalmente hay al menos dos resúmenes: un resumen en el mismo idioma que el texto principal y otro resumen en algún otro idioma.

El orden de los resúmenes varía según la escuela. Si su escuela tiene regulaciones específicas con respecto al orden de los resúmenes, la plantilla \gls{novathesis} (\LaTeX) las respetará. De lo contrario, la regla por defecto en la plantilla \gls{novathesis} es tener en primer lugar el resumen en \emph{el mismo idioma que el texto principal}, y luego el resumen en \emph{el otro idioma}. Por ejemplo, si la disertación está escrita en portugués, el orden de los resúmenes será primero en portugués y luego en inglés, seguido del texto principal en portugués. Si la disertación está escrita en inglés, el orden de los resúmenes será primero en inglés y luego en portugués, seguido del texto principal en inglés.
%
Por ejemplo, para un documento principal escrito en alemán con resúmenes escritos en alemán, inglés e italiano (en este orden) use:

\begin{verbatim}
    \ntsetup{abstractorder={<LANG_1>,...,<LANG_N>}}
    \ntsetup{abstractorder={<MAIN_LANG>={<LANG_1>,...,<LANG_N>}}}
\end{verbatim}

En cuanto a su contenido, los resúmenes no deben exceder una página y pueden responder a las siguientes preguntas (es fundamental adaptarse a las prácticas habituales de su área científica):
\begin{verbatim}
    \ntsetup{abstractorder={de={de,en,it}}}
\end{verbatim}

Per quanto riguarda i suoi contenuti, gli abstract non devono superare una pagina e possono rispondere alle seguenti domande (è essenziale adeguarsi alle pratiche abituali della propria area scientifica):

\begin{itemize}
\item Qual è il problema?
\item Perché è interessante?
\item Qual è la soluzione?
\item Quello che segue dalla soluzione?
\end{itemize}

% Palavras-chave do resumo em Inglês
\begin{keywords}
Parole chiave 1, Parole chiave 2, Parole chiave 3
\end{keywords}
